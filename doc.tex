\documentclass[twocolumn]{article}
\usepackage[a4paper, margin=0.7in]{geometry}
\usepackage{float}
\usepackage{amsmath}
\usepackage{amsfonts}
\usepackage{lipsum}
\usepackage{mathtools}
\usepackage{cuted}
\usepackage{pgf}
\usepackage{subfigure}
\usepackage{listings}
\usepackage[T1]{fontenc}
\usepackage[polish]{babel}
\usepackage[utf8]{inputenc}
\author{Grzegorz Janysek}
\title{Dokumentacja rozwiązania}

\renewcommand{\arraystretch}{1.25}
\setlength{\columnsep}{26pt}
\lstset{
  	aboveskip=10pt,
  	belowskip=18pt,
	frame=tb,
	columns=flexible,
	basicstyle={\ttfamily},
	breaklines=true,
	breakatwhitespace=true,
	tabsize=2,
	inputencoding=utf8,
}

\begin{document}

\maketitle

\section{Problem i rozwiązanie}
% Opis problemu i opis rozwiązania (w pseudokodzie)
Celem jest wygenerowanie grafu reprezentującego labirynt, a następnie wyświetlenie wizualizacji tego grafu.
Labirynt ma nie zawierać cykli i ma być spójny.
Aby to zapewnić znajdywane jest losowe drzewo rozpinające graf początkowy będący siatką prostokątną wierzchołków.

Jako argumenty program otrzymuje rozmiar generowanego labiryntu: \(n\) kolumn na \(m\) wierszy.
\begin{lstlisting}
fn main(szerokosc, wysokosc) {
	graf = wygeneruj_graf(szerokosc, wysokosc);
	znajdz_drzewo_rozpinajace(graf);
	wizualizuj(graf);
}
\end{lstlisting}

Na podstawie argumentów tworzony jest graf nieskierowany o \(n*m\) wierzchołkach i krawędziach pomiędzy każdą parą sąsiadujących wierzchołków.
W sposób taki, aby każdy z wierzchołków poza tymi które znajdują sie z krawędzi prostokątnej siatki miał czterech sąsiadów: na górze, na dole, po lewej i po prawej stronie.
\begin{lstlisting}
fn wygeneruj_graf(szerokosc, wysokosc) {
	graf = Graph::new();

	for i in 0..(szerokosc * wysokosc) {
		// dodaj wierzcholek o grupie 'i'
		graf.graph.add_node(i);
	}

	// dodaj poziome krawedzie
	for x in 1..szerokosc {
		for y in 0..wysokosc {
			graf.add_edge(
				node_at(x, y),
				node_at(x - 1, y),
				// oznacz krawedz jako
				// nie nalezaca do drzewa
				false,
			)
		}
	}

	// dodaj poziome krawedzie
	for x in 0..szerokosc {
		for y in 1..wysokosc {
			graf.add_edge(
				node_at(x, y),
				node_at(x, y - 1),
				// oznacz krawedz jako
				// nie nalezaca do drzewa
				false,
			)
		}
	}

	return graf
}
\end{lstlisting}

Następnie znajdywane jest pseudolosowe drzewo rozpinające graf.
Użyty do tego został zmodyfikowany algorytm Kruskala, różniący się wyborem krawędzi.
Algorytm iteruje po krawędziach grafu w losowej kolejności.
Jeżeli natrafi na krawędź łączącą wierzchołki w różnych grupach tj. należących do rozdzielnych poddrzew grafu: łączy te wierzchołki w jedno drzewo scalając grupy do których należą i oznaczając krawędź jak należącą do drzewa.
Oznaczone w ten sposób krawędzie tworzą zakończeniu iteracji drzewo rozpinające graf.
\begin{lstlisting}
fn znajdz_drzewo_rozpinajace(graf) {
	// zapewnia losowa kolejnosc itreacji
	edges = graf.edges().shuffle()

	for e in edges {
		// jezeli krawedz laczy 
		// wierzcholki o roznych grupach
		if e.a.grupa != e.b.grupa {
			// zlacz te grupy...
			merge(e.a.grupa, e.b.grupa)
			// ...i oznacz krawedz
			// jako nalezaca do drzewa
			e.value = true
		}
	}
}
\end{lstlisting}

\section{Użyte struktury danych}
% Opis użytych struktur danych
W programie wykorzystano grafu jako struktury przechowującej wierzchołki labiryntu.
Implementacja grafu używa macierzy sąsiedztwa. Pozwala ona na reprezentację grafów skierowanych oraz nieskierowanych.
Struktura jest generyczna i umożliwia na parametryzowanie typów wag krawędzi jak i wierzchołków.
Zaimplementowano podstawowe metody CRUD, iteratory i referencje dla wierzchołków i krawędzi grafu.

\section{Oszacowanie złożoności grafu}
% Oszacowanie złożoności czasowej i pamięciowej użytych struktur danych i podstawowych operacji na tych strukturach danych

Złożoność pamięciowa ograniczona jest rozmiarem macierzy sąsiedztwa i wynosi \(O(n^2)\).
Złożoności czasowe operacji na grafie kształtują się następująco:
\begin{table}[H]
	\centering
	\begin{tabular*}{\columnwidth}{l @{\extracolsep{\fill}} c}
		\hline
		operacja                & złożoność \\
		\hline\hline
		dodanie wierzchołka     & \(O(n)\) \\
		odczytanie wierzchołka  & \(O(1)\) \\
		zmiana wagi wierzchołka & \(O(1)\) \\
		usunięcie wierzchołka   & \(O(n)\) \\
		\hline
		dodanie krawędzi        & \(O(1)\) \\
		odczytanie krawędzi     & \(O(1)\) \\
		zmiana wagi krawędzi    & \(O(1)\) \\
		usunięcie krawędzi      & \(O(1)\) \\
		\hline
		odczytanie sąsiadów wierzchołka		& \(O(n)\) \\
		odczytanie wierzchołków krawędzi	& \(O(1)\) \\
		\hline
	\end{tabular*}
	% \caption{}
\end{table}

\section{Oszacowanie złożoności zmodyfikowaego algorytmu Kruskala}
% Oszacowanie złożoności czasowej i pamięciowej głównych algorytmów wykorzystanych w projekcie
\(n\) jest ilością wierzchołków.
Operacja utworzenia listy krawędzi o losowej kolejności działa w czasie \(O(n)\).

Porównanie grup wierzchołki odbywa się w czasie \(O(1)\).

Scalenie grup wierzchołków dla dedykowanej do tego zadania struktury zbiorów rozłącznych ma złożoność czasową \(O(log\;n)\).
W omawianej implementacji nie jest ona jednak wykorzystywana co pogarsza złożoność tej operacji do \(O(n)\). Jest to wynikiem sposobu łączenia grup \(a\) i \(b\), polegającego na iteracji po wszystkich wierzchołkach \(V\) takich że grupa \(V\) to \(b\) a następnie zmiany grupy \(V\) na \(a\).

Porównanie oraz scalanie grup wymagane jest dla każdej z krawędzi, których to liczba dąży do \(2n\). Daje to całkowitą złożoność czasową wynoszącą \(O(n^2)\)

Złożoność pamięciowa dyktowana jest rozmiarem (tworzonej w początkowym kroku) listy krawędzi i wynosi \(O(n)\).

\section{Dokumentacja użytkowa}
% Dokumentacja użytkowa: jak uruchomić program, jak wprowadzać dane
Opisane w dalszej części komendy wymagają łańcucha narzędzi języka Rust.

\subsection{Testowanie}
Uruchomienie zautomatyzowanych testów biblioteki zawierającej implementację grafu:
\begin{lstlisting}
	$ cargo test
\end{lstlisting}

\subsection{Kompilacja}
Kompilacja biblioteki grafu oraz programu generującego labirynt.
Plik wykonywalny znajduje się pod ścieżką \verb|target/release/maze|:
\begin{lstlisting}
	$ cargo build --release
\end{lstlisting}

\subsection{Uruchomienie programu}
Uruchomienie program generującego labirynt z domyślnymi argumentami:
\begin{lstlisting}
	$ cargo run --release
\end{lstlisting}
Wyświetlenie pomocy do programu:
\begin{lstlisting}
	$ cargo run --release -- -w 30 -h 5
\end{lstlisting}
Uruchomienie programu z argumentami określającymi wielkość labiryntu:
\begin{lstlisting}
	$ cargo run --release -- -w 30 -h 5
\end{lstlisting}

\end{document}